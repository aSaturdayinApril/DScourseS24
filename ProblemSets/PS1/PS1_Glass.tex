\documentclass{article}

% Language setting
% Replace `english' with e.g. `spanish' to change the document language
\usepackage[english]{babel}

% Set page size and margins
% Replace `letterpaper' with `a4paper' for UK/EU standard size
\usepackage[letterpaper,top=2cm,bottom=2cm,left=3cm,right=3cm,marginparwidth=1.75cm]{geometry}

% Useful packages
\usepackage{amsmath}
\usepackage{graphicx}
\usepackage[colorlinks=true, allcolors=blue]{hyperref}

\title{Your Paper}
\author{You}

\begin{document}
\maketitle

\begin{abstract}
Your abstract.
\end{abstract}

\section{Introduction}

I have always held an interest in Economics; I like it as an intersection of mathematics, and as an applicable social science. During the COVID lock-downs, many places in the U.S. faced food shortages. It was these events that really prompted me to leave my sales job to pursue something that I believe could have a bigger and positive impact on my local and national community. My main focus in pursuing a Master's in Managerial Economics was to work in a position (possibly at the Department of Agriculture) that would serve to support the national food supply. Through my academic and research journey, I have also developed an interest in international trade and labor economics.

I have fumbled my way through datasets in the past, and I have seen that one can truly harness data to gain insights and convey a story. However, I am no master in this area, and I think improving my skills in how I access and work with data could truly be beneficial.

\(x^2 + y^2 = z^2\)



\bibliographystyle{alpha}
\bibliography{sample}

\end{document}