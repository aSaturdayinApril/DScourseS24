\documentclass{article}
\usepackage{graphicx} % Include the graphicx package for including images
\usepackage{listings} % Include the listings package for code listings

\begin{document} % Begin the document environment

\begin{lstlisting}[language=R, caption={Linear Regression Output}]
Call:
lm(formula = logwage ~ hgc + college + tenure + age + married, 
    data = wages)

Residuals:
     Min       1Q   Median       3Q      Max 
-1.80084 -0.23093  0.02974  0.24627  0.86314 

Coefficients:
                          Estimate Std. Error t value
(Intercept)              0.6385419  0.1458085   4.379
hgc                      0.0618155  0.0054274  11.389
collegenot college grad  0.1464113  0.0347664   4.211
tenure                   0.0233959  0.0016747  13.970
age                     -0.0006992  0.0027606  -0.253
marriedsingle           -0.0238215  0.0178513  -1.334
                        Pr(>|t|)    
(Intercept)             1.26e-05 ***
hgc                      < 2e-16 ***
collegenot college grad 2.68e-05 ***
tenure                   < 2e-16 ***
age                        0.800    
marriedsingle              0.182    
---
Signif. codes:  
0 ‘***’ 0.001 ‘**’ 0.01 ‘*’ 0.05 ‘.’ 0.1 ‘ ’ 1

Residual standard error: 0.3465 on 1663 degrees of freedom
  (560 observations deleted due to missingness)
Multiple R-squared:  0.1949,	Adjusted R-squared:  0.1925 
F-statistic: 80.51 on 5 and 1663 DF,  p-value: < 2.2e-16
\end{lstlisting}

\textbf{Question 6:} About 24.8 percent of the values in "logwage" are missing. This missing values are likely missing completely at random (MCAR), meaning that the missingness is unrelated of any unobserved data..

\textbf{Question 8:}

I have decided to use financial and stock market data to gain a better understanding of potential investments. The dataset I am using includes all stock tickers from the NASDAQ and NYSE, totaling around 5,600 stock tickers.

\begin{figure}
    \centering
    \includegraphics[width=0.99\linewidth]{Ds7.png} % Adjust the width value to make the image larger
    \caption{Enter Caption}
    \label{fig:enter-label}
\end{figure}

\end{document}
