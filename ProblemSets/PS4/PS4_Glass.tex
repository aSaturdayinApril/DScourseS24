\documentclass{article}
\usepackage{hyperref}

\title{Data Sources for Scraping}
\author{Your Name}
\date{\today}

\begin{document}

\maketitle

\section{Introduction}
In this document, I will discuss various data sources that I am interested in scraping for analysis. These sources provide a wide range of data that can be valuable for research, analysis, and other purposes.

\section{Data Sources}
\subsection{Project Gutenberg}
\href{https://www.gutenberg.org/}{Project Gutenberg} offers a vast collection of classical texts that are freely available for download. These texts include novels, poems, plays, and other literary works from various time periods and cultures. Scraping data from Project Gutenberg can be useful for studying literature, language analysis, and cultural trends.

\subsection{IPUMS API}
\href{https://www.ipums.org/}{IPUMS (Integrated Public Use Microdata Series)} offers an API that allows users to access their extensive collection of harmonized census and survey data programmatically. The IPUMS API provides access to datasets from various countries and time periods. Researchers and developers can use this API to query specific variables, extract customized datasets, and integrate IPUMS data into their analyses.

\subsection{FRED API}
\href{https://fred.stlouisfed.org/docs/api/fred/}{FRED (Federal Reserve Economic Data)} provides an API for accessing a vast array of economic and financial data maintained by the Federal Reserve Bank of St. Louis. The FRED API allows users to retrieve time series data on various economic indicators such as GDP, inflation, unemployment, interest rates, and more. Researchers, analysts, and developers can use this API to access timely and comprehensive economic data for their projects and analyses.

\end{document}